\documentclass[14pt,a4paper]{article}
\usepackage[T2A]{fontenc}
\usepackage[utf8]{inputenc}
\usepackage[russian]{babel}
\usepackage{cmap}
\usepackage{amsmath,amssymb,amsthm,mathrsfs,graphicx,epstopdf,placeins,geometry,multirow,hhline}
\usepackage{hyperref}


\newcommand{\Tr}{\mathsf{\scriptstyle T}}
\def\XX{\mathbb{X}}
\def\YY{\mathbb{Y}}
\def\RR{\mathbb{R}}
\def\LL{{\mathscr L}}
\def\cL{\mathscr{L}}

\title{ICFP Contest 2017}
\author{DNIWE :: a}
\date{August 2017}

\begin{document}

\maketitle

\section{Оценка рек}

Обозначим через $ A_G \in \{0, 1\}^{N \times N} $ --- матрицу смежности для текущего графа $ G $ с $ N $ рёбер (рек).
Через $ X \in \RR^{N \times D} $ --- набор признаков ребра (его вес, принадлежит ли оно текущему игроку, соединяет ли с шахтой, и т.п.).

Для простой оценки привлекательности реки $ e $ можно использовать функцию $ f(e) = X_e^\Tr W $ (простое скалярное произведение двух векторов), где $ W \in \RR^{D} $  --- настраиваемые параметры.

\section{Учёт соседей}

Учитывать признаки соседей в модели можно просто добавив в качестве признака сумму признаков соседей, что достигается простым перемножением матрицы смежности и текущего набора признаков $ A_G X $.
Умножив $ A_G $ на получившуюся матрицу, получим учём вторых соседей, и так далее (циклы, к сожалению, будут вносить свою лепту в оценку).

Эти добавочные признаки обозначим через $ P $.

\section{Нелинейности}

Простая линейная модель имеет сильно меньше предсказательной силы.
К счастью добавление нелинейности в нашу модель жутко простое, и это можно сделать даже не в одном месте.

\subsection{Изменение признакового пространства}

Добавив в качестве признаков попарные произведения уже существующих, мы перенесём модель в другое бОльшее пространство, где будет проще решать задачу.

В итоге признаки $ X $ превратятся в $ X' $, и все штуки выше применяются уже к $ X' $.

\subsection{Нелинейность в модели}

У нас в модели есть конкатенация вида $ \phi(x) = g(g(x)) $, где $ g(x) $ --- линейная функция.
В такие вещи очень просто добавить нелинейность, просто впухнув нелинейную функцию активации $ \psi(x) $ (например, ReLU)\footnote{\url{https://en.wikipedia.org/wiki/Rectifier_(neural_networks)}}, изменив таким образом функцию $ g(x) $: $ g'(x) = \psi(g(x)) $.


\end{document}
